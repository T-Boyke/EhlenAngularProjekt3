\documentclass[12pt, a4paper, titlepage]{scrartcl}

% -------------------------------------------------------------------
% P R E A M B L E
% -------------------------------------------------------------------

% Encoding & Language
\usepackage[utf8]{inputenc}
\usepackage[T1]{fontenc}
\usepackage[ngerman]{babel}

% Font: Helvetica (Arial-like) is standard for IHK
\usepackage{helvet}
\renewcommand{\familydefault}{\sfdefault}
\usepackage{setspace}
\onehalfspacing % 1.5 line spacing

% Margins (IHK Standard: Left 2.5cm, Right 2cm, Top/Bottom 2.5cm)
\usepackage[left=2.5cm, right=2.0cm, top=2.5cm, bottom=2.5cm]{geometry}

% Graphics
\usepackage{graphicx}
\usepackage{float}

% Tables
\usepackage{tabularx}
\usepackage{booktabs}
\usepackage{longtable}

% Colors
\usepackage{xcolor}
\definecolor{codegray}{rgb}{0.5,0.5,0.5}
\definecolor{codepurple}{rgb}{0.58,0,0.82}
\definecolor{backcolour}{rgb}{0.95,0.95,0.92}
\definecolor{ihkblue}{RGB}{0,51,102}

% Code Listings
\usepackage{listings}
\lstdefinelanguage{TypeScript}{
  keywords={typeof, new, true, false, catch, function, return, null, catch, switch, var, if, in, while, do, else, case, break, const, let, class, export, boolean, throw, implements, import, this, interface, public, private, protected, readonly, number, string, any, void, extends, constructor},
  keywordstyle=\color{ihkblue}\bfseries,
  ndkeywords={class, export, boolean, throw, implements, import, this},
  ndkeywordstyle=\color{darkgray}\bfseries,
  identifierstyle=\color{black},
  sensitive=false,
  comment=[l]{//},
  morecomment=[s]{/*}{*/},
  commentstyle=\color{codegray}\ttfamily,
  stringstyle=\color{codepurple}\ttfamily,
  morestring=[b]',
  morestring=[b]"
}

\lstset{
  language=TypeScript,
  backgroundcolor=\color{backcolour},   
  basicstyle=\ttfamily\footnotesize,
  breakatwhitespace=false,         
  breaklines=true,                 
  captionpos=b,                    
  keepspaces=true,                 
  numbers=left,                    
  numbersep=5pt,                  
  showspaces=false,                
  showstringspaces=false,
  showtabs=false,                  
  tabsize=2,
  frame=single
}

% Hyperlinks
\usepackage[hidelinks]{hyperref}

% Header & Footer
\usepackage{scrlayer-scrpage}
\clearpairofpagestyles
\ihead{Tobias Boyke}
\chead{Earth Ocean Learning}
\ohead{\today}
\cfoot{\pagemark}
\pagestyle{scrheadings}

% Commands for placeholder images
\newcommand{\placeholderImage}[2]{
    \begin{figure}[H]
        \centering
        \fbox{\begin{minipage}{0.8\textwidth}
            \centering
            \vspace{2cm}
            \textbf{Bildplatzhalter: #1} \\
            (Bitte Bilddatei einfügen: #2)
            \vspace{2cm}
        \end{minipage}}
        \caption{#1}
        \label{fig:#2}
    \end{figure}
}

% -------------------------------------------------------------------
% D O C U M E N T
% -------------------------------------------------------------------

\begin{document}

% --- Deckblatt ---
\begin{titlepage}
    \centering
    \includegraphics[width=0.4\textwidth]{logo_placeholder.png} % Platzhalter für BitLC/Firmenlogo
    \vspace{1cm}
    
    {\scshape\LARGE Abschlussprojekt Dokumentation \par}
    \vspace{0.5cm}
    {\Large Fachinformatiker für Anwendungsentwicklung \par}
    \vspace{1.5cm}
    
    {\huge\bfseries Earth Ocean Learning APP \par}
    \vspace{0.5cm}
    {\Large Entwicklung einer kindgerechten Angular SPA zur Vermittlung von Wissen über die 5 Ozeane \par}
    
    \vspace{2cm}
    
    \begin{tabular}{ll}
        \textbf{Prüfling:} & Tobias Boyke \\
        \textbf{Prüfungsnummer:} & 0001278495 \\
        \textbf{Anschrift:} & Musterstraße 1, 40470 Düsseldorf \\
        & \\
        \textbf{Ausbildungsbetrieb:} & Beispiel GmbH \\
        \textbf{Anschrift:} & Firmenweg 2, 41460 Neuss \\
        & \\
        \textbf{Abgabedatum:} & 09.12.2025 \\
    \end{tabular}
    
    \vfill
    
    \emph{IHK Mittlerer Niederrhein (Neuss)}
\end{titlepage}

% --- Verzeichnisse ---
\pagenumbering{Roman} % Römische Seitenzahlen für Verzeichnisse

\tableofcontents
\newpage
\listoffigures
\listoftables
\lstlistoflistings
\newpage

% --- Glossar ---
\section*{Glossar}
\addcontentsline{toc}{section}{Glossar}
\begin{longtable}{p{0.25\textwidth}p{0.7\textwidth}}
    \toprule
    \textbf{Begriff} & \textbf{Erklärung} \\
    \midrule
    SPA & Single Page Application - Eine Webanwendung, die aus einer einzigen HTML-Seite besteht. Inhalte werden dynamisch nachgeladen. \\
    Angular & TypeScript-basiertes Open-Source-Framework von Google zur Erstellung skalierbarer Webanwendungen. \\
    Standalone Component & Architekturkonzept in Angular, bei dem Komponenten ohne NgModules auskommen. \\
    MVVM & Model-View-ViewModel - Architekturmuster zur Trennung von GUI und Logik. \\
    CI/CD & Continuous Integration / Continuous Delivery - Automatisierte Tests und Bereitstellung. \\
    JSON & JavaScript Object Notation - Kompaktes Datenformat. \\
    CMS & Content Management System. \\
    Tailwind CSS & Utility-First CSS-Framework. \\
    TypeScript & Statisch typisierte Erweiterung von JavaScript. \\
    WCAG & Web Content Accessibility Guidelines - Standard für Barrierefreiheit. \\
    LocalStorage & Web API zur dauerhaften Speicherung von Daten im Browser. \\
    \bottomrule
\end{longtable}
\newpage

% --- Inhalt ---
\pagenumbering{arabic} % Arabische Seitenzahlen für den Inhalt

\section{Einleitung}

\subsection{Ausgangssituation}
Im Rahmen der Umschulung zum Fachinformatiker für Anwendungsentwicklung erläutert der Autor sein Abschlussprojekt bei der fiktiven Firma \textbf{Beispiel GmbH}. Das Unternehmen ist ein junges IT-Unternehmen in Neuss, spezialisiert auf Webanwendungen und CMS-Lösungen. Mit 14 Mitarbeitern bedient es Kunden aus Edutainment, Logistik und Produktion.
Die Beispiel GmbH wurde von einer Umweltschutz-NGO (vertreten durch Herrn Dr. Uwe Umwelt) beauftragt, eine Browser-Applikation für das spielerische Erlernen von Ozean-Fakten für Grundschulkinder zu entwickeln .

\subsection{Projektidea und Zielsetzung}
Ziel ist die Entwicklung der Single Page Application (SPA) "Earth Ocean Learning". Kernfunktionen umfassen Ozean-Auswahl, Lernbereich, Quiz-Modul und Fortschrittsspeicherung. Die App muss als Standalone-Modul in CMS-Systeme integrierbar sein .

\subsection{Projektbegründung}
Moderne Webtechnologien (Angular, Signals) sollen herkömmliche Lernmaterialien ablösen, um Kindern einen modernen Anreiz zu bieten. Das Projekt dient zudem als Referenz im Bereich "Edutainment" .

\subsection{Make-or-Buy Entscheidung}
Es wurde eine Eigenentwicklung ("Make") gegenüber Standardlösungen ("Buy") bevorzugt. Gründe:
\begin{itemize}
    \item \textbf{Datenschutz:} Kein Tracking/Login (No-Tracking) für Kinder gefordert.
    \item \textbf{Offline-Fähigkeit:} Muss auch ohne stabiles WLAN laufen (PWA-Vorbereitung).
    \item \textbf{Kosten:} Einmalkosten günstiger als langfristige Lizenzgebühren .
\end{itemize}

\section{Projektplanung}

\subsection{Ist-Analyse}
Es existiert keine Softwarelösung ("Greenfield Project"). Inhalte liegen unstrukturiert vor. Es wird eine Standard-Entwicklerumgebung (Windows 11/Rocky Linux, VS Code) genutzt .

\subsection{Soll-Analyse}
\textbf{Funktionale Anforderungen:}
\begin{itemize}
    \item Ozean-Auswahl und Fakten-Anzeige (Carousel).
    \item Quiz-System mit direktem Feedback.
    \item Lokale Speicherung des Fortschritts (Sterne-System).
    \item Master-Quiz (Unlock nach Abschluss aller Ozeane).
\end{itemize}
\textbf{Nicht-funktionale Anforderungen:}
\begin{itemize}
    \item Kindgerechtes UI/UX.
    \item Technologie: Angular 21 (Signals), Standalone Components.
    \item Performance und Responsive Design .
\end{itemize}

\subsection{Zeitplanung}
Der Projektzeitraum ist vom 17.11.2025 bis 09.12.2025.

\placeholderImage{Visualisierung der Zeitplanung mittels Gantt-Diagramm}{gantt_chart}

\begin{table}[H]
\centering
\caption{Zeitplanung}
\begin{tabularx}{\textwidth}{lXr}
\toprule
\textbf{Phase} & \textbf{Tätigkeit} & \textbf{Zeit (h)} \\
\midrule
\textbf{1. Analyse \& Planung} & & \textbf{11 h} \\
& Ist-Analyse \& Soll-Konzept & 3 h \\
& Pflichtenheft / Fachkonzept & 4 h \\
& Wirtschaftlichkeit / Ressourcen & 4 h \\
\midrule
\textbf{2. Entwurf} & & \textbf{12 h} \\
& UI/UX Design (Mockups) & 5 h \\
& Architektur \& Datenmodell & 4 h \\
& Auswahl Tools & 3 h \\
\midrule
\textbf{3. Implementierung} & & \textbf{28 h} \\
& Setup Umgebung & 2 h \\
& Core-Komponenten \& Routing & 6 h \\
& Logik (QuizService) & 8 h \\
& UI \& Styling & 8 h \\
& Daten-Integration & 4 h \\
\midrule
\textbf{4. Qualitätssicherung} & & \textbf{9 h} \\
& Testfälle \& Bugfixing & 9 h \\
\midrule
\textbf{5. Dokumentation} & & \textbf{10 h} \\
\midrule
\textbf{Gesamt} & & \textbf{70 h} \\
\bottomrule
\end{tabularx}
\end{table}

\subsection{Kostenplanung}
Basierend auf einem internen Stundensatz von 35,00 € (Azubi inkl. GK):
\begin{itemize}
    \item Personalkosten: 70 h * 35,00 € = 2.450,00 €
    \item Sachmittel (Entwicklungsumgebung): 150,00 €
    \item \textbf{Gesamtkosten (Plan): 2.600,00 €} 
\end{itemize}

\section{Analyse \& Entwurf}

\subsection{Anwendungsfalldiagramm (Use Cases)}
\placeholderImage{Anwendungsfalldiagramm (Use Cases)}{usecase_diagram}

\subsection{Architekturentwurf}
Die Anwendung ist eine reine SPA (Client-Side Rendering) ohne Server-Side Rendering (SSR), um Hosting-Kosten zu sparen und App-Feel zu gewährleisten.
Die Architektur orientiert sich am MVVM-Muster unter Nutzung von Angular Signals (Fine-Grained Reactivity) und JSON als Datenquelle .

\placeholderImage{Architekturentwurf (View Layer, ViewModel, Data Layer)}{architecture_diagram}

\subsection{UI/UX Design}
Design-Fokus: Helle Farben, große Buttons, wenig Text, Barrierefreiheit (WCAG) .
\textit{(Siehe Anhang für Mockups)}

\subsection{Datenmodell}
Die Datenhaltung erfolgt dokumentenorientiert via JSON.

\placeholderImage{Hierarchische Struktur des JSON-Objekts}{datamodel_diagram}

\begin{lstlisting}[caption={Beispiel Datenblock (JSON)}, label={lst:json}]
{
  "id": "pacific",
  "name": "Pazifischer Ozean",
  "quizQuestions": [
    {
      "question": "Wie tief ist der Marianengraben?",
      "options": ["11.000m", "5.000m", "2.000m"],
      "correctAnswer": "11.000m"
    }
  ]
}
\end{lstlisting}

\subsection{Klassendiagramm}
Das Klassendiagramm zeigt die Abhängigkeiten zwischen Components und dem zentralen \texttt{QuizService}.

\placeholderImage{Klassendiagramm (UML)}{class_diagram}

\subsection{Datenschutz \& Sicherheit}
Privacy by Design: Keine PII-Erhebung, lokale Speicherung (LocalStorage), kein Tracking .

\section{Realisierung}

\subsection{Entwicklungsumgebung}
Tools: Visual Studio Code, GitHub, Angular CLI v21, Google Chrome .

\subsection{Implementierung der Hauptkomponenten}

\subsubsection{Standalone Components}
Verzicht auf NgModules zur Reduktion der Komplexität.

\begin{lstlisting}[caption={Standalone Component Definition}, label={lst:standalone}]
@Component({
  selector: 'app-ocean-facts',
  standalone: true,
  imports: [CommonModule, NgOptimizedImage],
  templateUrl: './ocean-facts.component.html',
  styleUrl: './ocean-facts.component.css',
  changeDetection: ChangeDetectionStrategy.OnPush
})
export class OceanFactsComponent { ... }
\end{lstlisting}

\subsubsection{State Management via QuizService}
Nutzung des "Service with Signals"-Patterns.

\begin{lstlisting}[caption={Service with Signals (Auszug)}, label={lst:signals}]
@Injectable({ providedIn: 'root' })
export class QuizService {
  // Private State
  private _score = signal<number>(0);
  private _completedOceans = signal<string[]>([]);
  
  // Public API
  readonly score = this._score.asReadonly();
  
  // Mutation
  addStar(oceanId: string) {
    this._completedOceans.update(list => [...list, oceanId]);
  }
}
\end{lstlisting}

\subsection{Herausforderungen \& Lösungen}
\begin{table}[H]
\caption{Herausforderungen \& Lösungen}
\begin{tabularx}{\textwidth}{lX}
\toprule
\textbf{Herausforderung} & \textbf{Lösung} \\
\midrule
Persistierung des Fortschritts & Nutzung eines Angular \texttt{effect} im Service, der \texttt{localStorage} automatisch updated. \\
Linting Inkompatibilität & Verzicht auf ESLint zugunsten strikter Compiler-Checks. \\
Kindgerechte UX & Große Icons, intuitive Farben, einfache Navigation. \\
\bottomrule
\end{tabularx}
\end{table}

\section{Qualitätssicherung}

\subsection{Testplanung}
Fokus auf manuellen Black-Box-Tests und funktionalen Systemtests. Automatisierte Unit-Tests (Karma/Jest) wurden aufgrund von Versionskonflikten in Angular 21 verworfen, aber beispielhaft evaluiert .

\begin{lstlisting}[caption={Beispiel Unit-Test}, label={lst:test}]
it('should calculate correct score', () => {
  const service = TestBed.inject(QuizService);
  service.answerQuestion(true);
  service.answerQuestion(true);
  expect(service.score()).toBe(2);
});
\end{lstlisting}

\subsection{Testdurchführung}
Alle kritischen Testfälle (TF01 - TF06) waren erfolgreich.
\begin{table}[H]
\centering
\begin{tabularx}{\textwidth}{llXl}
\toprule
ID & Testfall & Erwartung & Status \\
\midrule
TF01 & App-Start & Startbildschirm lädt & OK \\
TF03 & Quiz-Logik & Feedback Grün/Rot & OK \\
TF04 & Fortschritt & Speicherung LocalStorage & OK \\
TF05 & Master-Quiz & Unlock bei 5 Sternen & OK \\
\bottomrule
\end{tabularx}
\end{table}

\subsection{Google Lighthouse Audit}
Performance: 80/100 (wegen PNGs), Accessibility: 100/100, Best Practices: 100/100 .

\section{Wirtschaftlichkeitsbetrachtung}

\subsection{Soll-Ist-Vergleich (Zeit \& Kosten)}
Die geplante Zeit von 70h wurde genau eingehalten. Abweichungen im Entwurf (+2h) wurden durch schnellere QS (-1h) und Doku (-1h) ausgeglichen.
Die tatsächlichen Kosten belaufen sich auf 2.600,00 € .

\subsection{Amortisationsrechnung}
Vergleich: Druckkosten Broschüren (Analog) vs. App-Betrieb (Digital).
\begin{itemize}
    \item Kosten Analog (Druck/Logistik): 2.000,00 € / Jahr
    \item Kosten Digital (Hosting/Wartung): 260,00 € / Jahr
    \item Jährliche Einsparung: 1.740,00 €
\end{itemize}

\begin{equation}
Amortisationszeit = \frac{\text{Projektkosten}}{\text{Jährliche Einsparung}} = \frac{2.600,00}{1.740,00} \approx 1,49 \text{ Jahre}
\end{equation}

Das Projekt amortisiert sich nach ca. 18 Monaten .

\section{Fazit \& Ausblick}
Das Projekt wurde erfolgreich umgesetzt ("in Time", "in Budget"). Die Entscheidung für Angular Signals und Standalone Components hat sich bezüglich Performance und Code-Qualität ausgezahlt, trotz "Early-Adopter"-Hürden bei Tooling (Linter/Tests).
\textbf{Ausblick:} Geplante Features für V2.0 sind Sound-Effekte, PWA-Support und Mehrsprachigkeit (i18n).
Am 09.12.2025 erfolgte die Abnahme ohne Mängel .

\newpage
\appendix
\section{Anhang}

\subsection{Verzeichnisstruktur}
\begin{verbatim}
/
├── src/
│   ├── app/
│   ├── assets/
│   └── index.html
├── angular.json
├── package.json
└── README.md
\end{verbatim}

\subsection{Mockups und Screenshots}
% Hinweis: Hier müssten die Bilder aus der Quelle [317-321] eingefügt werden.
\placeholderImage{Startbildschirm (Skizze vs. Umsetzung)}{mockup_start}
\placeholderImage{Quiz-Bildschirm}{mockup_quiz}
\placeholderImage{Ergebnis-Bildschirm}{mockup_results}

\subsection{Code-Beispiele (QuizService)}
\begin{lstlisting}[caption={Vollständiger QuizService Logic}, label={lst:fullservice}, basicstyle=\tiny\ttfamily]
import { Injectable, computed, inject, signal, effect } from '@angular/core';
import { DataService } from '../services/data.service';
import { Ocean } from '../models/ocean.model';

@Injectable({ providedIn: 'root' })
export class QuizService {
  private dataService = inject(DataService);
  private readonly STORAGE_KEY = 'eol_progress';

  // State Signals
  private _oceans = signal<Ocean[]>([]);
  private _currentOceanId = signal<string | null>(null);
  private _score = signal<number>(0);
  private _completedOceans = signal<string[]>([]);
  private _masterMode = signal<boolean>(false);

  readonly oceans = this._oceans.asReadonly();
  readonly isMasterUnlocked = computed(() => this._completedOceans().length >= 5);

  constructor() {
    const saved = localStorage.getItem(this.STORAGE_KEY);
    if (saved) this._completedOceans.set(JSON.parse(saved));
    
    effect(() => {
       localStorage.setItem(this.STORAGE_KEY, JSON.stringify(this._completedOceans()));
    });
  }

  startMasterQuiz() {
    // Logic to mix all questions
    this._masterMode.set(true);
    // ... Implementation details
  }
}
\end{lstlisting}

\subsection{Abnahmeprotokoll}
\textbf{Datum:} 09.12.2025 \\
\textbf{Ergebnis:} Mängelfrei abgenommen. \\
\textbf{Unterschriften:} Tobias Boyke (Auftragnehmer), Dr. Uwe Umwelt (Auftraggeber).

\newpage
\section{Erklärung}
Hiermit versichere ich, dass ich die vorliegende Arbeit selbstständig und ohne fremde Hilfe angefertigt und keine anderen als die angegebenen Quellen und Hilfsmittel verwendet habe.

\vspace{3cm}
\noindent
\begin{tabular}{lp{2cm}l}
\cline{1-1} \cline{3-3}
Ort, Datum & & Unterschrift \\
Neuss, 08.12.2025 & & Tobias Boyke
\end{tabular}

\end{document}
